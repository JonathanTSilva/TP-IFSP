%==============================================================
%------------------JONATHAN TOBIAS DA SILVA--------------------
%---------------ENGENHARIA ELÉTRICA - 1ª TURMA-----------------
%-----------INSTITUTO FEDERAL - CAMPUS SERTÃOZINHO-------------
%--------------Main.tex, v1.0.0, JonathanTSilva----------------
%==============================================================

\documentclass[../Main.tex]{subfiles}

\begin{document}

    Os objetivos devem ser claros e diretos, mencionando as informações sobre o que se pretende estudar.
    
    Procura-se responder às seguintes perguntas para elaborar os objetivos:
    
    \begin{enumerate}[a)]
        \item Que perguntas específicas este estudo procura responder?
        \item Quais hipóteses serão testadas?
        \item Para que? Para quem?
        \item Quais são os objetivos gerais?
    \end{enumerate}
    
    Os objetivos devem ser elaborados com verbos mais precisos que indicam sentido único de interpretação.
    
    Exemplo de verbos \textbf{\underline{mais}} precisos: discutir, identificar, relacionar, construir, comparar, traduzir, integrar, selecionar, ilustrar, interpretar, distinguir, resumir, classificar, enumerar, aplicar, resolver, localizar, confeccionar, assinalar, escrever, indicar, descrever, caracterizar, elaborar, encaminhar, instrumentalizar, capacitar, formular, propor, intervir, participar, investigar, verificar, questionar e qualificar.
    
    Exemplo de verbos \textbf{\underline{menos}} precisos: aprender, conhecer, apreciar, pensar, compreender, entender, valorizar, tolerar, respeitar, familiarizar-se, desejar, acreditar, saber, avaliar, desfrutar, temer, interessar, motivar, captar, orientar, aumentar, melhorar, conscientizar, estimular, reconhecer, acertar e refletir.

\end{document}