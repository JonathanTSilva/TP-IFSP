%==============================================================
%------------------JONATHAN TOBIAS DA SILVA--------------------
%---------------ENGENHARIA ELÉTRICA - 1ª TURMA-----------------
%-----------INSTITUTO FEDERAL - CAMPUS SERTÃOZINHO-------------
%--------------Main.tex, v1.0.0, JonathanTSilva----------------
%==============================================================

\documentclass[../Main.tex]{subfiles}

\begin{document}
    \begin{minipage}{0.23\linewidth}
    	\begin{figure}[H]
    		\includegraphics[scale=0.30]{Images/logo-ifsp}
    	\end{figure}
    \end{minipage}
    \begin{minipage}{0.39\linewidth}
    	\begin{table}[H]
    	\def\arraystretch{1} %esp. entre linhas
    		\begin{tabular}{c}
    			\textbf{JONATHAN TOBIAS DA SILVA} \\
    			jonathantobias2009@hotmail.com \\
    			Engenharia Elétrica \\
    			Instituto Federal de São Paulo
    		\end{tabular}
    	\end{table}
    \end{minipage}
    \begin{minipage}{0.26\linewidth}
    	\begin{figure}[H]
    		\flushright
    		\includegraphics[scale=0.07]{Images/logo-eng2}
    	\end{figure}
    \end{minipage}
    
    \vspace{0.5cm}
    
    \mysection{NOTAS}
        \begin{enumerate}
            \item {\color{red} ESTA SEÇÃO DEVERÁ SER EXCLUÍDA DO SEU DOCUMENTO FINAL.}
            
            \item Este documento foi desenvolvido em \LaTeX\ como \textit{template} do Relatório Parcial necessário para o acompanhamento de um projeto de Iniciação Científica do Instituto Federal de Educação, Ciência e Tecnologia do Estado de São Paulo - IFSP - Campus Sertãozinho.
            
            \item Este documento está disponível no repositório do GitHub:
            
            \begin{center}
                \href{https://github.com/JonathanTSilva}{github.com/JonathanTSilva}
            \end{center}
            
            \item Todos os campos que deverão ser alterados estão apresentados na forma de colchetes: [CAMPO].
            
            \item Para a exclusão desta seção, ir para o arquivo principal  (Main.tex) e deletar as seguintes linhas:
            
\begin{verbatim}
%=======================EXCLUIR============================
\label{sec:README}
\subfile{Sections/README}
\newpage
%==========================================================
\end{verbatim}
        \end{enumerate}
         
    \mysection{HISTÓRICO DE MODIFICAÇÕES}
        A \autoref{tab:modificacoes} apresenta o histórico de modificações deste documento, que deverão ser anotadas e descritas as versões desenvolvidas deste \textit{template}, acrescentando na tabela o respectivo número de versão, a elaboração, aprovação e breve descrição da alteração realizada.
        
        \begin{table}[H]
            \centering
            \caption{Histórico de modificações}
            \begin{tabular}{|>{\centering\arraybackslash}m{0.07\textwidth}|>{\centering\arraybackslash}m{0.25\textwidth}|>{\centering\arraybackslash}m{0.25\textwidth}|>{\centering\arraybackslash}m{0.3\textwidth}|}
                \hline
                \textbf{V.}         & \textbf{ELABORAÇÃO}                   & \textbf{APROVAÇÃO}                    & \textbf{DESCRIÇÃO} \\ \hline
                \multirow{3}{*}{1.0.0}  & \hrulefill                            & \hrulefill                            & {\scriptsize Modelo de relatório -} \\
                                        & {\scriptsize Jonathan T. da Silva}    & {\scriptsize Jonathan T. da Silva}    & {\scriptsize Criação da \textit{template} em} \LaTeX \\
                                        & {\scriptsize 30/Jul, 2020}            & {\scriptsize 30/Jul, 2020}            & {\scriptsize } \\ \hline
                                        
                                        
                                        
                                        
                \multirow{3}{*}{}       & \hrulefill                            & \hrulefill                            & {\scriptsize [DESCRIÇÂO LINHA1]} \\
                                        & {\scriptsize [NOME DO ELABORADOR]}    & {\scriptsize [NOME DO APROVADOR]}     & {\scriptsize [DESCRIÇÂO LINHA2]} \\
                                        & {\scriptsize [DATA DA ELABORAÇÃO]}    & {\scriptsize [DATA DA APROVAÇÃO]}     & {\scriptsize [DESCRIÇÂO LINHA3]} \\ \hline
            \end{tabular}
            \label{tab:modificacoes}
        \end{table}
        
\end{document}